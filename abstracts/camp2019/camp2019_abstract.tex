\documentclass[11pt]{article}
\usepackage[T1]{fontenc}
\usepackage[scaled]{uarial}
\usepackage{graphicx}
\renewcommand*\familydefault{\sfdefault} 
\usepackage[margin=1in]{geometry}

\begin{document}
\thispagestyle{empty}
\begin{center} \textbf{EXPLAINING AWAY AND SEMANTIC ADAPTATION}  \\
\end{center}
\vspace{-1em}
\noindent \textbf{Keywords}: explaining away, adaptation, semantics, pragmatics, computational psycholinguistics


\noindent Speakers exhibit considerable production variability at all levels of linguistic representation. Listeners deal with such variability by adapting to it and updating expectations [1-4]. The research on semantic/pragmatic adaptation has been extended to show that listeners adapt to varying uses of the uncertainty expressions \textit{might} and \textit{probably} (\emph{\textbf{``You might / You'll probably} get a blue gumball''})[5]. Using this information, they proposed a novel model within the Rational Speech Acts Model. However, on this model, it was poorly understood how speakers evaluate when they should or shouldn't update their beliefs. There is evidence, for example, that speakers may use quantifiers and modals differently for reasons that have to do with external contextual factors [6, 7]. We further develop the research on modals, "might" and "probably" to explore how speakers use information from the context to determine when they should or shouldn't adapt, result. We explore the possibility of an explaining away effect, where become less likely to update their expectations about a speaker in the face of contextual information that explains their behavior [8].

\noindent \textbf{Exp.~1 (norming)} tested whether mood was a reliable factor for influencing the expectations for the use of "might" and "probably".  53 MTurk participants saw 36 images of an airline representative interacting with a customer, along with a seat map that illustrated the likelihood of the customer getting their desired seat. Participants in the \textit{optimist} and \textit{pessimist} conditions saw an airline representative who was in a good or bad mood respectively, while participants in the \textit{neutral} condition saw a representative with no information about mood given. The images ranged from a 0\% to 100\% chance of receiving the desired seat. Participants were then asked to rate utterances by distributing 100 points across three items, "You'll probably get one," "You might get one, " and a blanket, something else, option. As the ... figure shows, probably was rated higher than might for a greater range of possibilities in the \textit{optimistic} condition than in the \textit{pessimistic} condition. Like [4], we quantified this difference by fitting a spline for each expression and participant and then computing the area under the curve (AUC). We found that the average difference between the AUC of \textit{might} and \textit{probably} is smaller in the \textit{optimistic} cond than in the \textit{pessimistic} cond ($t(34)=-2.51$, $p < 0.05$), suggesting that mood affects expectations about the use of \textit{might} and \textit{probably}. 

\noindent \textbf{Exp.~2 (explaining away)} tested whether additional information about the mood affected the process of adaptation. 268 MTurk participants saw 13 exposure trials (5 critical, 8 fillers) followed by 36 test trials. Exposure trials showed an image of an airline representative describing a seat map with window and aisle seats (critical trials: 60\% window seats). There were 4 conditions - \textit{optimist}, \textit{confident}, \textit{pessimist}, \textit{cautious} - where participants in the first 2 were exposed to a speaker using \textit{probably} at 60\% and participants in the second 2 were exposed to a speaker using \textit{might}. Fillers were intended to boost trust in the speaker: on 5 trials, the speaker described a typical (25\% and 90\%) probability with the respective other uncertainty expression. On the other fillers, the speaker said of a 100\% window seat flight “You’ll get one”. Additionally, for the exposure in the \textit{optimist} and \textit{pessimist} conditions, the participants were given extra information about the speaker's mood, she was happy or angry respectively. The test trials were the same as in Exp. 1, and the participants were all told the flight attendant was now in a normal mood. We found that the average difference between the AUC of \textit{might} and \textit{probably} is smaller in the \textit{confident} cond than in the \textit{cautious} ($t(133)=-5.1755$, $p < 0.001$), suggesting suggesting adaptation of expectations about might and probably use. Further, we found that the average difference between the AUC of \textit{might} and \textit{probably} is smaller in the \textit{pessimist} cond than in the \textit{cautious} ($t(135)=-2.38$, $p < 0.05$)and qualitatively smaller in the \textit{confident} cond than in the \textit{optimist}, suggesting that users engaged in explaining away.


\noindent \textbf{Conclusion}: We showed that semantic/pragmatic adaptation takes advantage of information in the context to determine how much listeners should adapt to their speaker. 


\noindent \textbf{References:} 

\noindent $[$1$]$ Kleinschmidt, D. \& Jaeger, T.F. (2016). Robust speech perception: Recognize the familiar, generalize to the similar, and adapt to the novel. \textit{Psychological Review}.
\\ $[$2$]$ Kamide, Y. (2012). Learning individual talkers' structural preferences. \textit{Cognition}.
\\$[$3$]$~Fine, A.B., Jaeger, T.F., Farmer, T.A. \& Qian, T. (2013). Rapid expectation adaptation during syntactic comprehension. \textit{PLOS ONE}.
\\$[$4$]$ Yildirim, I., Degen, J., Tanenhaus, M.K. \& Jaeger, T.F. (2016). Talker-specificity and adaptation in quantifier interpretation. \textit{Journal of Memory and Language}.
\\$[$5$]$  Schuster, S. \& Degen, J. (2019). Speaker-specific adaptation to variable use of uncertainty expressions. Under review.
\\$[$6$]$   Bonnefon, J.F., Feeney, A. \& Villejoubert, G. (2007). When some is actually all: Scalar inferences in face-threatening contexts. \textit{Cognition}.
\\$[$7$]$  Juanchich, M. \& Sirota, M. (2013). Do people really say it is "likely" when they believe it is only "possible"? Effect of politeness on risk communication.
\\$[$8$]$  Wellman, M.P. \& Henrion, M. (1992). Explaining "Explaining Away". \textit{IEEE Transactions on Pattern Analysis and Machine Translation}.
\\$[$9$]$ Goodman, N.D. \& Frank, M.C. (2016). Pragmatic language interpretation as probabilistic inference. \textit{Trends in Cognitive Science}.
\end{document}